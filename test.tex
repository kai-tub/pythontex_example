\documentclass{article}
\usepackage{pythontex}
\usepackage{amsmath}
\newcommand{\pymultiply}[2]{\py{#1*#2}}

% https://jeltef.github.io/PyLaTeX/current/usage.html
% https://github.com/gpoore/pythontex

\begin{document}
So this is how I got everything up and running.
First of all, I am using \emph{latexmk} from VS Code.
Because of restrictions from PythonTex I am forced to
either write the generated python files into the current
directory so that \emph{latexmk} finds them if the output directory
is changed, or to keep the output directory for \emph{latexmk}
to the current directory, in which case the python files can be
moved to a different directory. I find the latter case nicer,
so I set the VS Code option to not generate the \emph{latexmk} 
files in a different folder. This may has to be changed
in the VS Code \texttt{settings.json}.

Then to utilize the build system from within VS Code and use
\emph{latexmk} correctly, I had to create the following
\texttt{.latexmkrc} file. I only include this file in the
current project directory, as I don't want this to be run, if
I am not using PythonTex.

{\small
\begin{verbatim}
add_cus_dep('pytxcode', 'tex', 0, 'pythontex');
# PYTHON_PATH = /home/kai/miniconda3/envs/pythontex/bin/python
# TEX3_PATH = /usr/share/texlive/texmf-dist/scripts/pythontex/pythontex3.py
# To find PYTHON_TEX3_PATH: cat $(which pythontex3)
sub pythontex { return system("PYTHON_PATH PYTHON_TEX3_PATH 
    --interpreter python:PYTHON_PATH \"$_[0]\""); }
\end{verbatim}
}

As one can see, I am hardcoding a lot. But the main reason for this
is that I wanted to use an isolated python environment and
I couldn't get it to work otherwise. But with this, the
python environment is standalone and doesn't interfere with the
main one.

I do like the functionality to create Python snippets in latex, and
to do some integral math within a \LaTeX\ document, but I wanted
to also print some matrices, with the ability to mutate them
in a similar manner. For this I am also using the
\emph{pylatex} python package, which can create \LaTeX\ snippets
for matrices, tables, and so on.


\begin{pycode}
print("Python says ``Hello''")
\end{pycode}

$8 \times 256 = \pymultiply{8}{256}$

\begin{pycode}
from pylatex import Matrix, Math
import numpy as np
M = (np.random.rand(3, 3) * 10).round(2)
matrix = Matrix(M, mtype="b")

math = Math(data=["M=", matrix])
print(math.dumps())
\end{pycode}

\end{document}